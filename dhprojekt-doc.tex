% !TEX encoding = UTF-8
\documentclass[a4paper,12pt,titlepage=true, ngerman]{scrartcl}
%\usepackage{ngerman}, 
%\usepackage[ngerman]{babel}

\usepackage[utf8]{inputenc}
\usepackage[ngerman]{babel}
\usepackage[babel]{csquotes}

\usepackage
	[backend=biber, style=authoryear-comp, maxbibnames=3,
	 isbn=false, doi=false, eprint=false, dashed=false]{biblatex}
\ExecuteBibliographyOptions{maxcitenames=2}
\bibliography{literature.bib}
\usepackage{hyperref}
%\hypersetup{
%colorlinks=true, linktocpage=false, pdfborder={0 0 0}, pdfstartview=FitV, 
%urlcolor=Black, linkcolor=Black, citecolor=Black, %pdfstartpage=3, 
%}
\usepackage{graphicx}

\usepackage{scrhack}
\KOMAoptions{BCOR=8mm}
%\KOMAoptions{toc=flat}
% \KOMAoptions{toc=graduated, toc=bibliography}

% \renewcommand*\sectfont{\rmfamily\mdseries\scshape}
\setkomafont{section}{\rmfamily\bfseries\LARGE}
% \setkomafont{sectionentry}{\rmfamily\bfseries\scshape\Large}
\setkomafont{subsection}{\rmfamily\bfseries\Large}
\setkomafont{subsubsection}{\rmfamily\bfseries\large}
%\setkomafont{chapter}{\rmfamily\bfseries\scshape\huge}
%  \setkomafont{partentry}{\rmfamily\bfseries\scshape}
% \setkomafont{chapterentry}{\rmfamily\bfseries\scshape}
\setkomafont{partentry}{\rmfamily\bfseries\scshape}
\setkomafont{part}{\rmfamily\bfseries\scshape\huge}
\setkomafont{partnumber}{\rmfamily\bfseries\scshape\huge}
\setkomafont{partentrypagenumber}{\rmfamily\bfseries\scshape}

\setkomafont{descriptionlabel}{\bfseries}

%  scrhack to fuck float error message

%%%%  for page head and foot %%%%%%%
\usepackage[automark]{scrpage2}
\pagestyle{scrheadings}
\KOMAoptions{headsepline=true}
\setheadsepline{.4pt}
\setkomafont{pageheadfoot}{\normalfont\normalcolor\upshape} 

\setcounter{tocdepth}{3}
\setcounter{secnumdepth}{4}

\usepackage[T1]{fontenc} % to make Sa\"ibi searchable
\usepackage[protrusion=true,expansion=true]{microtype}
% \usepackage{microtype}
% \usepackage{setspace}\setstretch{1.2}



% \usepackage{lmodern} 


%  \usepackage{palatino}\linespread{1.05}
 \usepackage{mathpazo}\linespread{1.05}  
\usepackage[scaled=.95]{helvet}
   \usepackage{courier}
   \usepackage[scaled]{berasans}
 
% \usepackage[bitstream-charter]{mathdesign}  %%%%%%%%%%%%%%%%%%%%%  FINAL VERSION %%%%%%%%%%%%%%%%%%%%%%%%%%%%%%%

%\usepackage[style=authoryear,
% 	maxnames=1,
%  	maxbibnames=99,                  %%%%%%%%     TODO  activate
% uniquename=init
%]{biblatex}

\usepackage{calc}
\usepackage{ifthen}

\usepackage{tikz}
\usetikzlibrary{shapes,arrows}

\tikzstyle{task} = [diamond, draw, fill=green!20, 
    text width=4.5em, text badly centered, node distance=2cm, inner sep=0pt]
\tikzstyle{comp} = [rectangle, draw, fill=blue!20, 
    text width=5em, text centered, rounded corners, minimum height=4em, node distance=2cm]
\tikzstyle{line} = [draw, -latex']
\tikzstyle{line2} = [draw, double, -latex']
\tikzstyle{human} = [draw, ellipse,fill=red!20, text width=5em, text centered, node distance=2cm,
    minimum height=3em]
\tikzstyle{decision} = [diamond, draw, fill=green!20, 
    text width=4.5em, text badly centered, node distance=3cm, inner sep=0pt]
\tikzstyle{block} = [rectangle, draw, fill=blue!20, 
    text width=5em, text centered, rounded corners, minimum height=4em]
\tikzstyle{cloud} = [draw, ellipse,fill=red!20, text width=5em, text centered, node distance=3cm,
    minimum height=2em]

% % Pie chart
\newcommand{\slice}[4]{
  \pgfmathparse{0.5*#1+0.5*#2}
  \let\midangle\pgfmathresult

  % slice
  \draw[thick,fill=black!10] (0,0) -- (#1:1) arc (#1:#2:1) -- cycle;

  % outer label
  \node[label=\midangle:#4] at (\midangle:1) {};

  % inner label
  \pgfmathparse{min((#2-#1-10)/110*(-0.3),0)}
  \let\temp\pgfmathresult
  \pgfmathparse{max(\temp,-0.5) + 0.8}
  \let\innerpos\pgfmathresult
  \node at (\midangle:\innerpos) {#3};
}

\usepackage{bchart}


\usepackage{setspace}
\onehalfspacing

% \usepackage{framed}

\usepackage{graphicx}

\usepackage{ownstyle}

\usepackage{cleveref}

\usepackage{footnote}


%###########################################################################
%%%%%%%%%%%%%%%%%%%%%%%%%%%%%%%%%%%%%%%%%%%%%%%%%%%%%%%%
%###########################################################################

\pagenumbering{roman}

\title{Dokumentation zur Arbeit von Gruppe 2 im Projekt ''Praxis der Digital Humanities''} % title
\author{Clemens Ahrens}

\begin{document}
\begin{titlepage}

\begin{center}

\vspace*{100pt}

\textbf{\Large{Dokumentation zur Arbeit von Gruppe 2 im Projekt ''Praxis der Digital Humanities''}}% title

\vfill

Schriftliche Dokumentation

im Seminar

\emph{Praxis der Digital Humanities}


Leitung:

\textbf{Prof.\ Dr.\ Caroline Sporleder}% leitung

WiSe 2014/2015% semester

\bigskip
\bigskip

an der Universität Trier

Fachbereich II

\bigskip
\bigskip

Verfasser:


\textbf{Clemens Ahrens}

Matrikelnummer: 1116125

\textbf{Andre} %TODO

Matrikelnummer: xxxxxx

\textbf{Christopher Michels} %TODO

Matrikelnummer: 1007830

\bigskip
\bigskip

24. Februar 2015

\vfill

\end{center}

\end{titlepage}


%###########################################################################
%%%%%%%%%%%%%%%%%%%%%%%%%%%%%%%%%%%%%%%%%%%%%%%%%%%%%%%%
%###########################################################################

\subsection*{Eidesstattliche Erklärung}
Hiermit erkläre wir, dass wir die Hausarbeit selbstständig
verfasst und keine anderen als die angegebenen Quellen und Hilfsmittel benutzt
und die aus fremden Quellen direkt oder indirekt übernommenen Gedanken als
solche kenntlich gemacht haben.

Die Arbeit haben wir bisher keinem anderen Prüfungsamt in gleicher oder
vergleichbarer Form vorgelegt. Sie wurde bisher nicht veröffentlicht.


\vspace{3cm}
\begin{center}
24. Februar 2015 \hspace{7.5cm} Clemens Ahrens
\\ \hspace{10.5cm} André Beyer
\\ \hspace{10.5cm} Christopher Michels
\end{center}
% \left Datum
% \right Unterschrift


%###########################################################################
%%%%%%%%%%%%%%%%%%%%%%%%%%%%%%%%%%%%%%%%%%%%%%%%%%%%%%%%
%###########################################################################

\newpage

\microtypesetup{protrusion=false}
 \tableofcontents%[liststotoc]
\microtypesetup{protrusion=true}

% \newpage
% \listoffigures
% \listoftables


%###########################################################################
%%%%%%%%%%%%%%%%%%%%%%%%%%%%%%%%%%%%%%%%%%%%%%%%%%%%%%%%
%###########################################################################

\newpage

\listoffigures


%###########################################################################
%%%%%%%%%%%%%%%%%%%%%%%%%%%%%%%%%%%%%%%%%%%%%%%%%%%%%%%%
%###########################################################################

\newpage

\listoftables

%###########################################################################
%%%%%%%%%%%%%%%%%%%%%%%%%%%%%%%%%%%%%%%%%%%%%%%%%%%%%%%%
%###########################################################################

\newpage
\pagenumbering{arabic}

\section{Einleitung}\label{Einleitung} %TODO
 
Im Rahmen des Seminars \emph{Praxis der Digital Humanities} haben sich die einzelnen Gruppen vorgenommen, ein Werkzeug zu entwickeln, das literaturwissenschaftliche Arbeit unterstützt. Diese Hilfssoftware soll dazu dienen, die stereotype Darstellung von einzelnen Personen oder Gruppen in Texten zu analysieren und zu visualisieren. Vor diesem Hintergrund baut die hier vorgestellte Arbeit der zweiten Gruppe auf den Vorverarbeitungsschritten der ersten Gruppe auf und fügt für die drei Werke \emph{Uncle Tom's Cabin}  \autocite[]{chris_uncle}, \emph{To Kill a Mockingbird}  \autocite[]{chris_bird} und \emph{The Adventures of Huckleberry Finn: Tom Sawyer} \autocite[]{chris_adventures} \textbf{Informationen zu Koreferenzketten} hinzu. Auf diese Weise trägt die Gruppe zwei neben der \emph{Named Entity Recognition} als Vorverarbeitungsschritt der ersten Gruppe dazu bei, dass die Protagonisten erkennbar werden und so von der dritten Gruppe auf ihre literarische Darstellung hin untersucht werden können.

Im Folgenden werden \emph{Bart 2.0}, \emph{DCoref} und \emph{Reconcile} als Softwarewerkzeuge vorgestellt, die zunächst für die Koreferenzresolution in Betracht gezogen wurden. Dieser Ausgangspunkt stellte die zweite Gruppe vor die Aufgabe, die einzelnen Werkzeuge zu bewerten und zu vergleichen. Nach der Zusammenfassung des Verlaufs und der Ergebnisse dieses Vergleichs wird dann kurz dargestellt, welche Formatierungsanforderungen für die Schnittstellen zu den anderen Gruppen beachtet wurden und wie kapitelbezogene Koreferenzketten zu kapitelübergreifenden Ketten zusammengeführt wurden. Zuletzt folgt ein Beispiel für die Verwendung und die Ausgabe eines der Werkzeuge.


%###########################################################################
%%%%%%%%%%%%%%%%%%%%%%%%%%%%%%%%%%%%%%%%%%%%%%%%%%%%%%%%
%###########################################################################

\newpage

\section{Ansätze zur Koreferenzresolution}\label{Ansätze zur Koreferenzresolution} %TODO


%###########################################################################

\subsection{BART 2.0}

 \section{clemens\_text}\label{clemens_text}
 
 bla
 



%###########################################################################

\subsection{DCoref}

% !TEX encoding = UTF-8
Das Werkzeug \emph{DCoref} ist ein deterministisches Modul zur Koreferenzresolution und ist Teil der \emph{Stanford CoreNLP}. Für die Verwendung dieses Moduls ist vorausgesetzt, dass andere Module vorher auf die Eingabedaten angewendet worden sind \autocite[]{chris_stanford_dcoref}. Die Module für Wortart- und Lemmabestimmung sowie Named Entitiy Recognition und für einen Parser müssen als sogenannte Annotatoren beim Verwenden der \emph{Stanford CoreNLP} wie in Listing \ref{dcoref:required_annotators} angebeben werden.

\begin{lstlisting}[label=dcoref:required_annotators, language=Java, caption=Angabe vorausgesetzter Annotatoren in Java]
//	configure properties for pipeline of the Stanford CoreNLP
Properties properties = new Properties();
properties.setProperty("annotators", 
		"pos, lemma, ner, parse, dcoref");
//	[...] set more properties and instantiate pipeline
StanfordCoreNLP pipeline = new StanfordCoreNLP(properties);
\end{lstlisting}

% ist deterministisches Modul der Stanford CoreNLP
%hasngt von anderen Modulen ab (pos, lemma, ner, parse)
%diese Module msussen in der Pipeline vorher angewendet werden 
%arbeitet mit Sieben
%erste Stufe: bevorzugt Recall (detection)
%zweite Stufe: Siebe bevorzugen Precision
%Post-Processing: mehr Precision (?)


%###########################################################################

\subsection{Reconcile} %TODO


%~~~

\subsubsection{Steckbrief} %TODO


%~~~

\subsubsection{Probleme} %TODO


%###########################################################################
%%%%%%%%%%%%%%%%%%%%%%%%%%%%%%%%%%%%%%%%%%%%%%%%%%%%%%%%
%###########################################################################

\newpage

\section{Verlauf der Gruppenarbeit}\label{Verlauf der Gruppenarbeit} %TODO


%###########################################################################

\subsection{Vergleich}%TODO


%~~~

\subsubsection{Ausgangssituation}\label{Ausgangssituation}

Wie bereits erwähnt, wurde \emph{BART 2.0} für den Vergleich der Ansätze zur Koreferenzresolution als Kandidat verworfen. Übrig geblieben sind also \emph{DCoref} und \emph{Reconcile}. Das Ziel des Vergleichs war es, mit einem MUC-Scorer anhand der Werte Precision, Recall und F-Score den besseren der verbleibenden zwei Kandidaten zu finden. Um einen Goldstandard für den Vergleich der beiden Ansätze zu erhalten, wurde die erste Hälfte des ersten Kapitels aus \emph{Uncle Tom's Cabin} \autocite[]{chris_uncle} als Testdatensatz für eine manuelle Annotierung mit Unterstützung des dafür vorgesehenen Softwarewerkzeugs \emph{MMAX} \autocite*[]{chris_mmax} verwendet \autocite[]{chris_mmax_coll}. Diese manuelle Annotierung sollte durch zwei Gruppenmitglieder durchgeführt und dann verglichen und verbessert werden, bevor der daraus resultierende Goldstandard für die Bewertung der beiden Ansätze verwendet wurde.


%~~~

\subsubsection{Probleme mit MMAX}%TODO

Die Verwendung von \emph{MMAX} hat bereits von Anfang an zu Problemen geführt. Trotz der vorhandenen Dokumentation war es schwierig, \emph{MMAX} zu verwenden. Beide Annotatoren hatten zunächst den Eindruck, das Werkzeug nicht richtig zu verwenden. Ein Beispiel dafür zeigt Abbildung \ref{mmax:anzeige}.

\begin{figure}[ht]
\begin{center}
\includegraphics[width=15cm]{./img/cmich/cm_mmax.jpg}
 % bart_webUI_output.png: 819x724 pixel, 72dpi, 28.89x25.54 cm, bb=0 0 819 724
\caption{MMAX - Beispiel der Anzeige}
\label{mmax:anzeige}
\end{center}
\end{figure}

Darin ist zu sehen, wie mithilfe von \emph{MMAX} versucht wurde, eine Koreferenzkette für den Protagonisten \emph{Haley} in \emph{Uncle Tom's Cabin} \autocite[]{chris_uncle} zu erstellen. Die Hervorhebung der Kette in gelber Farbe kann über ein mit einem Rechtsklick erreichbares Kontextmenü aktiviert werden, allerdings findet diese Hervorhebung nur in dem im aktuellen Fensterausschnitt sichtbaren Teil des Textes statt. Die türkise Farbe war ursprünglich als Markierungsfarbe für die Kette eines bestimmten Protagonisten gedacht. Dieses Vorhaben hat jedoch nicht funktioniert, da jedes einzelne Wort im Text mit türkiser Farbe markiert wurde. Bei der Erstellung des \emph{MMAX}-Projekts für die Annotierung konnte man verschiedenen Annotierungsebenen unterschiedliche Farben zuordnen, jedoch war es nicht ohne Weiteres möglich, diese Farben sinnvoll einzusetzen, wie das Beispiel in der Abbildung zeigt. Außerdem haben mit der Zeit länger werdende Ketten vermutlich ab einer gewissen Länge Speicherprobleme verursacht und zu korrupten Daten geführt, die wiederholte Abstürze für \emph{MMAX} zur Folge hatten. 

Insgesamt haben diese Probleme dazu geführt, dass es nicht sinnvoll erschien, mehr Zeit in die Suche nach Fehlern und Alternativen bei der Konfiguration eines \emph{MMAX}-Projektes zu investieren. Das vorläufige Ziel des Vergleichs der mit \emph{MMAX} erstellten Ketten wurde folglich als praktisch unmöglich eingeschätzt. Aus diesem Grund hat sich die Frage ergeben, wie das Ziel, einen Goldstandard zu erhalten, auf einem anderen Weg zu erreichen ist.


%~~~

\subsubsection{Alternative Lösung}%TODO

TODO: Standardweg: CoNLL

Als alternativen Lösungsweg hin zu einem Vergleich der Precision-, Recall- und F-Score-Werte für \emph{DCoref} und \emph{Reconcile} hat sich dann die  für die Testdaten angeboten. Diese Ausgabe wurde manuell von zwei Annotatoren korrigiert, um einen Goldstandard zu erhalten. Dabei wurde das \emph{Reconcile}-Ausgabeformat beibehalten. Dann wurde die \emph{DCoref}-Ausgabe für die Testdaten ebenfalls in das \emph{Reconcile}-Ausgabeformat überführt, wie es in Listings  \ref{output:vergleich:decoref} und \ref{output:vergleich:reconcile} zu sehen ist. Dafür mussten lediglich die Informationen innerhalb des <coreferences>-Knotens im Reintext der Testdaten entsprechend als <NP>-Tags an den richtigen Stellen eingefügt werden.

\begin{lstlisting}[label=output:vergleich:decoref, name=vergleich_decoref.xml, language=xml, caption=Ausschnitt der \emph{DCoref}-Ausgabe für die Testdaten]
<?xml version="1.0" encoding="UTF-8" standalone="no"?>
<chapter id="1" title="In Which [...] Humanity">
  <sentences/>
  <coreferences>
    ...
    <coreference id="2">
      <mention representative="true">
        <sentence>1</sentence>
        <start>34</start>
        <end>35</end>
        <head>34</head>
        <text>Kentucky</text>
      </mention>
      ...
    </coreference>
    ...
  </coreferences>
</chapter>
\end{lstlisting}

\begin{lstlisting}[label=output:vergleich:reconcile, name=vergleich_reconcile.xml, language=xml, caption=Ausschnitt aus Listing \ref{output:vergleich:reconcile} im \emph{Reconcile}-Ausgabeformat]
... in <NP NO="8" CorefID="8">Kentucky</NP>. There were ...
\end{lstlisting}

\noindent Für die Auswertung hat ein MUC-Scorer sowohl die nicht nachbearbeitete \emph{Reconcile}-Ausgabe als auch die umformatierte \emph{DCoref}-Ausgabe mit dem manuell erstellten Goldstandard verglichen. Die Ergebnisse dieses Auswertung sind in Tabelle \ref{score:ergebnis} zu finden.

\begin{savenotes}
	\begin{table}[ht]
		\centering
  		\begin{tabular}{ l || c | c | c ||  r }
						& Reconcile 	& DCoref 	& DCoref (PP\footnote[1]{PP steht hier für das Post-Processing, das in der optionalen dritten Stufe von \emph{DCoref} stattfindet.}) 	& \\ \hline \hline
    			Precision 	& 0.99673 	& 0.74510	& 0.74837		& \\ \hline
    			Recall     	& 0.97917 	& 0.35775 	& 0.33861 		& \\ \hline
    			FScore    	& 0.98787  	& 0.48340 	& 0.46625 		& \\ \hline
  		\end{tabular}
  		\caption{Auswertung der Ansätze mit einem MUC-Scorer}
  		\label{score:ergebnis}
	\end{table}
\end{savenotes}

Offensichtlich schneidet \emph{Reconcile} viel besser ab als \emph{DCoref}. Dieses Ergebnis liegt wahrscheinlich auch daran, dass die \emph{Reconcile}-Ausgabe als Ausgangspunkt für die Annotierung verwendet wurde, da auf diese Weise der Goldstandard und die \emph{Reconcile}-Ausgabe nicht unabhängig voneinander entstanden sind und deshalb sehr ähnlich sind. Außerdem zeigt die Auswertung auch, dass die optionale dritte Stufe, die bei \emph{DCoref} für die Nachverarbeitung zuständig ist, durch das Entfernen von relevanten \emph{mentions} die Scores negativ beeinflusst, wie es von den Autoren auch beschrieben wurde \autocite[29, 32]{chris_leeetal}.

Da der hier vorgestellte Vergleich nicht optimal abgelaufen ist, wurde die Entscheidung getroffen, sowohl mit \emph{DCoref} als auch mit \emph{Reconcile} Koreferenzinformationen zu erschließen und die jeweilige Ausgabe wie von der dritten Gruppe gewünscht zu formatieren. Dabei wurden die beiden Ansätze allerdings nicht in ein einzelnes System integriert. Die Ausgabedaten wurden mithilfe von separaten Programmen erstellt und entsprechend umformatiert.


%###########################################################################

\subsection{Formatierung}\label{Verlauf:Formatierung}%TODO


%~~~

\subsubsection{DCoref}%TODO


Nach Absprache fügt die erste Gruppe der Projektarbeit bereits Koreferenzinformationen hinzu, da sie die \emph{Stanford CoreNLP} ebenfalls verwendet und bereits für jedes der drei für dieses Projekt ausgewählten Bücher laufen lässt. Trotzdem bietet das Java-Programm, das für das Hinzufügen von Koreferenzinformationen mithilfe von \emph{DCoref} zuständig ist, die Möglichkeit, die \emph{Stanford CoreNLP} mit diesem Modul laufen zu lassen. Allerdings ist dieses Programm aufgrund der Absprache mit der ersten Gruppe standardmäßig so eingestellt, dass nur noch die Umformatierung in das von der dritten Gruppe gewünschte Format ausgeführt wird und anschließend die Erstellung der kapitelübergreifenden Koreferenzketten ausgeführt wird (s. Abschnitt \ref{Indizierung}).

Listing \ref{input:decoref:plaintext} zeigt, dass Reintext als Eingabe erwartet wird, falls \emph{DCoref} noch Koreferenzinformationen hinzufügen muss, bevor die Ausgabe umformattiert wird. Der dafür gewählte Beispieltext stellt den eindeutigen Endpunkt der ersten Hälfte des ersten Kapitels im Buch \emph{Uncle Tom's Cabin} \autocite[]{chris_uncle} dar, der für die Testdaten vereinbart wurde (s. Abschnitt \ref{Ausgangssituation}).

\begin{lstlisting}[label=input:decoref:plaintext, name=decoref_in_plain.xml, language=xml, caption={Reintext, Eingabe für \emph{DCoref}}]
...
And the trader leaned back in his chair, and folded his arm, with an air of virtuous decision, apparently considering himself a second Wilberforce. 
\end{lstlisting}

\noindent Die Ausgabe \emph{DCoref} ist in Listing \ref{output:decoref:unformatiert} zu sehen. Bei diesem Format sind zwei Dinge negativ aufgefallen. Einerseits kommt der <coreference>-Tag auf zwei verschiedene Arten vor: Er wird nicht nur für den Sammelknoten verwendet, der die eigentlichen XML-Knoten für die einzelnen Ketten enthält, sondern auch für die Knoten der Koreferenzketten selbst. Deshalb hat die dritte Gruppe um eine Umbenennung der Sammelknoten-Tags von <coreference> in <coreferences> gebeten. Für die Knoten der Koreferenzketten wurde außerdem eine eindeutige Nummerierung in Form eines \emph{id}-Attributs gebeten. 

\begin{lstlisting}[label=output:decoref:unformatiert, name=decoref_out.xml, language=xml, caption={XML, Unformatierte Ausgabe von \emph{DCoref}}]
<?xml version="1.0" encoding="UTF-8"?>
<?xml-stylesheet href="CoreNLP-to-HTML.xsl" type="text/xsl"?>
<root>
  <document>
    <sentences/>
    <coreference>
      ...
      <coreference>
        <mention representative="true"/>
        ...
      </coreference>
      ...
    </coreference>
  </document>
</root>
\end{lstlisting}

\noindent Durch eine Umformatierung werden die von der dritten Gruppe beschriebenen Mängel behoben. Dadurch wird das ursprüngliche Ausgabeformat von \emph{DCoref} in das in Listing \ref{output:decoref:formatiert} gezeigte endgültige Format gebracht. Dieses Format ist ebenfalls gruppenintern für die Schnittstelle zum Programm für die Erstellung der kapitelübergreifenden Koreferenzketten festgelegt worden.   

\begin{lstlisting}[label=output:decoref:formatiert, name=decoref_out_formatted.xml, language=xml, caption={XML, Formatierte Ausgabe von \emph{DCoref}}]
<?xml version="1.0" encoding="UTF-8" standalone="no"?>
<chapter id="1" title="In Which [...] Humanity">
  <sentences/>
  <coreferences>
    ...
    <coreference id="1">
      <mention representative="true">
      ...
    </coreference>
    ...
  </coreferences>
</chapter>
\end{lstlisting}

\noindent Das Programm für \emph{DCoref} bietet auch die Möglichkeit, die optionale dritte Stufe des Moduls auszuführen. Die Dateien, die bei der Auswahl dieser Option erstellt werden, werden entsprechend gekennzeichnet bzw. in separaten Ordnern abgelegt, um sie von eventuell bereits vorhandenen anderen Dateien zu unterscheiden. Diese Unterscheidung betrifft die Ausgabedateien der ersten Gruppe für jedes einzelne Kapitel sowie die durch den Aufruf des Programms für die Erstellung kapitelübergreifender Ketten erstellte Indexdatei. Weitere Details zu der verwendeten Ordnerstruktur und zu dieser Unterscheidung befinden sich in den im Rahmen des Projektes verwendeten öffentlichen \emph{GitHub}-Repositories (s. Abschnitt \ref{Repositories}).


%~~~

\subsubsection{Reconcile}%TODO

%###########################################################################

\subsection{Indizierung}\label{Indizierung}%TODO



% \section{clemens\_text}\label{clemens_text}
 
 bla
 


%\section{clemens\_code}\label{clemens_code}
 bla 
 
 böa
 %TODO headings for lstlistings
\begin{lstlisting}[caption=Main-Klasse der BuildIndex,label=code:Coref, name=Coref.java] 
import java.util.ArrayList;
import java.util.Map;

public class Coref {
	ArrayList<String> inFiles;
	String outFile;

	// parser for inputs
	void parseArgs(String[] args) {
		Integer iIdx = null;
		Integer oIdx = null;
		inFiles = new ArrayList<String>();

		// input must come before output ---
		for (int i = 0; i < args.length; i++) {
			if (args[i].equals("-i")) {
				iIdx = new Integer(i);
			} else if (args[i].equals("-o")) {
				oIdx = new Integer(i);
			}
		}
		if (iIdx != null & oIdx != null) {
			outFile = args[oIdx + 1];
			for (int i = iIdx + 1; i < oIdx; i++) {
				inFiles.add(args[i]);
			}
		} else {
			System.out.println("Usage: " + "\n"
					+ "java Coref -i <XmlFile> -o <NewFile>" + "\n"
					+ "Example (Linux): " + "\n"
					+ "java Coref -i ../Input/*.xml -o ../output.xml");
			System.exit(-1);
		}
	}

	public static void main(String[] args) {
		Coref coref = new Coref();
		coref.parseArgs(args);

		OutputWriter outputWriter = new OutputWriter(coref.getOutFile());

		// for each input XML: Analyze, build Maps of coreferences, pass to output XML
		try {
			for (String inFile : coref.getInFiles()) {
				InputAnalyzer inputAnalyzer = new InputAnalyzer(inFile);

				Map<String, Coreference> coreferences = inputAnalyzer
						.extractCoreferences();

				outputWriter.addCoreferences(coreferences);
			}
		} catch (Exception e) {

			e.printStackTrace();
		}

		// create output XML
		outputWriter.writeToFile();

	}

	public ArrayList<String> getInFiles() {
		return inFiles;
	}

	public String getOutFile() {
		return outFile;
	}
}
\end{lstlisting} 

\begin{lstlisting}[caption=Coreference-Klasse,label=code:Coreference, name=Coreference.java]
public class Coreference {
	String text;
	String chapId;
	Integer corefId;

	public Coreference(String text, String chapId, Integer corefId) {
		this.text = text;
		this.chapId = chapId;
		this.corefId = corefId;
	}

	public String getText() {
		return text;
	}

	public void setText(String text) {
		this.text = text;
	}

	public String getChapId() {
		return chapId;
	}

	public void setChapId(String chapId) {
		this.chapId = chapId;
	}

	public Integer getCorefId() {
		return corefId;
	}

	public void setCorefId(Integer corefId) {
		this.corefId = corefId;
	}

}
\end{lstlisting}

\begin{lstlisting}[caption=InputAnalyzer-Klasse,label=code:InputAnalyzer, name=InputAnalyzer.java]
import java.io.File;
import java.io.IOException;
import java.util.ArrayList;
import java.util.HashMap;
import java.util.List;
import java.util.Map;

import org.jdom.Document;
import org.jdom.Element;
import org.jdom.JDOMException;
import org.jdom.input.SAXBuilder;

public class InputAnalyzer {

	File infile;
	List<String> listOfText;

	public InputAnalyzer(String filename) throws Exception {
		infile = new File(filename);
		if (!infile.exists()) {
			throw new Exception("File " + filename + " not found");
		}
	}

	public Map<String, Coreference> extractCoreferences() {
		Map<String, Coreference> map = new HashMap<String, Coreference>();
		listOfText = new ArrayList<>();

		// aufrufen des builders pro input
		SAXBuilder builder = new SAXBuilder();
		try {
			Document doc = builder.build(infile);

			// ---- Create list of <coreferences> & extract chapID
			Element root = doc.getRootElement();

			// extract argument chapter ID for output document
			String chapID = root.getAttribute("id").getValue();

			// jump to necessary level of sub elements
			Element corefs = root.getChild("coreferences");

			@SuppressWarnings("unchecked")
			List<Element> listcoref = corefs.getChildren("coreference");
			for (Element coref : listcoref) {
				String corefID = coref.getAttribute("id").getValue();

				// create list of mentions for each coreference chain
				@SuppressWarnings("unchecked")
				List<Element> listmention = coref.getChildren("mention");
				for (Element mention : listmention) {
					if (null == mention)
						continue;
					
					// search for "representative" mention
					String s = mention.getAttributeValue("representative");
					if (null == s || !s.equals("true"))
						continue;

					// extract XML element <text> for output document
					String text = mention.getChildText("text");

					// build new Coreference element with needed attributes &
					// turn corefID to Integer
					Coreference coreference = new Coreference(text, chapID,
							Integer.parseInt(corefID));
					map.put(text, coreference);

				}
			}
		} catch (JDOMException | IOException e) {
			// TODO Auto-generated catch block
			e.printStackTrace();
		} catch (NumberFormatException e) {
			System.err.println("Error while parsing corefID: \n"
					+ e.getMessage());
		}

		return map;

	}

}
\end{lstlisting}
\begin{lstlisting}[caption=OutputWriter-Klasse der BuildIndex,label=code:OutputWriter, name=OutputWriter.java]
import java.io.File;
import java.io.FileOutputStream;
import java.io.IOException;
import java.util.ArrayList;
import java.util.HashMap;
import java.util.Map;
import java.util.Set;

import org.jdom.Document;
import org.jdom.Element;
import org.jdom.output.Format;
import org.jdom.output.XMLOutputter;

public class OutputWriter {

	File outFile;
	Map<String, ArrayList<Coreference>> multiMap;

	public OutputWriter(String filename) {
		outFile = new File(filename);
		multiMap = new HashMap<String, ArrayList<Coreference>>();
	}

	public void addCoreferences(Map<String, Coreference> coreferences) {
		Set<String> keySet = coreferences.keySet();
		for (String key : keySet) {
			// create ArrayList if key not already existent
			if (!multiMap.containsKey(key)) {
				multiMap.put(key, new ArrayList<Coreference>());
			}
			// add coreference for key to ArrayList in multiMap for key
			multiMap.get(key).add(coreferences.get(key));
		}
	}

	public void writeToFile() {
		// create output element, which will be turned to output XML later
		Document outputdoc = new Document(new Element("root"));

		// write basic elements
		Element chains = new Element("chains");
		Element outroot = outputdoc.getRootElement();
		outroot.addContent(chains);
		// ---- create sub elements for each coref in outputdoc
		for (String key : multiMap.keySet()) {
			Element chain = new Element("chain");
			chains.addContent(chain);
			chain.setAttribute("text", key);

			for (Coreference coreference : multiMap.get(key)) {
				Element coref = new Element("coreference");
				Element chapter = new Element("chapter");
				Element id = new Element("id");

				chain.addContent(coref);
				coref.addContent(id);
				id.addContent(coreference.getCorefId().toString());
				coref.addContent(chapter);
				chapter.addContent(coreference.getChapId());

			}
		}

		// format output XML file
		XMLOutputter outp = new XMLOutputter();
		outp.setFormat(Format.getPrettyFormat());

		// ---- Write the complete result document to output XML file ----
		try {
			outp.output(outputdoc, new FileOutputStream(outFile));
		} catch (IOException e) {
			System.err.println("Error writing output file.");
			e.printStackTrace();
		}
	}
}
\end{lstlisting}
  %TODO

% \input{generierung} %TODO

% \input{normalisierung} %TODO

% \input{evaluierung} %TODO

% \input{ausblick} %TODO


%###########################################################################
%%%%%%%%%%%%%%%%%%%%%%%%%%%%%%%%%%%%%%%%%%%%%%%%%%%%%%%%
%###########################################################################

\newpage

\section{Zusammenfassung}%TODO


%###########################################################################

\subsection{Aufgabenverteilung}%TODO

Bei der Bestimmung der Kandidaten für die Auswahl eines geeigneten Ansatzes für die Koreferenzresolution hat jedes Gruppenmitglied einen Ansatz übernommen. Clemens Ahrens hat sich mit \emph{BART 2.0} beschäftigt, André Beyer mit \emph{Reconcile} und Christopher Michels mit \emph{DCoref}.

Während Clemens Ahrens ein Programm für die Erstellung kapitelübergreifender Koreferenzketten entwickelt hat und nachdem \emph{BART 2.0} als Kandidat verworfen wurde, haben sich André Beyer und Christopher Michels mit dem Vergleich von \emph{DCoref} und \emph{Reconcile} beschäftigt. Nach vergeblichen Versuchen, \emph{MMAX} und in die betrachteten Softwarewerkzeuge integrierten Scorer für diese Aufgabe zu verwenden, wurde ein Goldstandard im \emph{Reconcile}-Ausgabeformat erstellt. Dann hat André Beyer einen MUC-Scorer für das gleiche Ausgabeformat entwickelt. Um \emph{DCoref} ebenfalls mit diesem MUC-Scorer bewerten zu können, hat Christopher Michels die Ausgabe von \emph{DCoref} für die Testdaten entsprechend umformatiert.

Außerdem hat André Beyer ein Programm zur Umformatierung der Ausgabe von \emph{Reconcile} für das von der dritten Gruppe geforderte Ausgabeformat geschrieben. Analog hat Christopher Michels die Ausgabe von \emph{DCoref} für die Schnittstelle der dritten Gruppe umformatiert. Clemens Ahrens hat mithilfe von ersten funktionierenden Programmversionen unserer Gruppe vorläufige Testdaten für das Buch \emph{Uncle Tom's Cabin} \autocite[]{chris_uncle} an die dritte Gruppe weitergegeben, damit diese neben selbsterstellten Daten weiteres Testmaterial zur Verfügung hatte.

%###########################################################################

\subsection{Kommentar EVENTUELL?}\label{Kommentar}%TODO

TODO: Sowas wie eine kurze abschließende Bewertung vielleicht? Mir (Chris) ist nur nichts Gutes eingefallen und ich muss noch für eine Klausur lernen. Wenn ich noch unbedingt was selbst ändern soll, sagt mir bitte bescheid. Wenn niemand sonst will, lassen wir es weg ... 


%###########################################################################

\subsection{Repositories}\label{Repositories}%TODO

Die folgenden öffentlichen Repositories wurden auf \emph{GitHub} \autocite[]{chris_github} im Rahmen der Gruppenarbeit erstellt und verwendet. Dort sind weitere Details der erstellten Programme sowie Informationen zu deren Verwendung einsehbar.

\begin{itemize}
  \item Umwandler von \emph{Reconcile} zur Schnittstelle der dritten Gruppe:\par \url{https://github.com/beyeran/transform-reconcile}\vspace{0.3cm}
  \item MUC Scorer für die \emph{Reconcile}-Dateien:\par \url{https://github.com/beyeran/score-coref}\vspace{0.3cm}
  \item Umwandlung von \emph{DCoref} zur Schnittstelle der dritten Gruppe:\par \url{https://github.com/cmich/dh-projekt-gruppe2-dcoref}\vspace{0.3cm}
  \item \emph{DCoref} für die Eingabedaten:\par \url{https://github.com/Rostu/dh-Projekt-Gruppe1}\vspace{0.3cm}
  \item Erstellung der Indexdatei für kapitelübergreifende Koreferenzketten: \par \url{https://github.com/ClemensAhrens/BuildIndex}\vspace{0.3cm}
\end{itemize}


%###########################################################################
%%%%%%%%%%%%%%%%%%%%%%%%%%%%%%%%%%%%%%%%%%%%%%%%%%%%%%%%
%###########################################################################

\newpage

% \printshorthands[title=Abkürzungsverzeichnis]
\printbibliography[title=Literaturverzeichnis, heading=bibintoc]


\end{document}