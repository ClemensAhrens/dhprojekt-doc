\subsubsection{Steckbrief}

Reconcile ist eine Art Baukasten für Koreferenzresolution. Reconcile soll es möglich machen, verschiedene Ansätze und Komponenten der Koreferenzresolution zu kombinieren und zu testen. Die Verarbeitung ist in fünf Schritte eingeteilt, welche modular und einfach anpassbar gestaltet sein sollen: Vorverarbeitung, Merkmalserzeugung, Klassifikation, Clustering und Scoring. Zur Vorverarbeitung zählen Schritte wie z.B. Zerteilung in Paragraphe, Sätze und Tokenizierung, POS-Tagging, Parsing und Named Entity Recognition. Ein Merkmal, innerhalb der Merkmalserzeugung, ist ein Substantivpaar, welches sich eine gemeinsame Eigenschaft teilt. Zwei Substantive, die im Numerus gleich sind, wären z.B. ein Merkmal. Bei der Klassifikation werden dann zwei Substantive anhand dieser Merkmale verglichen. Das Ergebnis ist ein Wert, der ausdrücken soll, wie wahrscheinlich es ist, dass beide Substantive koreferent sind. Über diese Werte wird ein Clusteringalgorithmus angewandt, um Ketten innerhalb der Koreferenzen festzustellen. Abschließend soll man mit einer implementierten Scoringfunktion das Ergebnis mit einem Goldstandart vergleichen können \autocite{reconcile2}.

Die Standartkonfiguration nutzt größtenteils \emph{Stanford OpenNLP} Werkzeuge für die Vorverarbeitung. Die hier verwendete Konfiguration ist identisch mit \autocite{reconcile}. Die Entwickler von Reconcile behaupten, es sei auf dem neusten Stand der Technik (\emph{state-of-the-art}).

\subsubsection{Probleme}
Das Hauptproblem ist eindeutig die mangelnde Dokumentation. Die Möglichkeiten zur Konfiguration und zum Testen von verschiedenen Modulen kann leider nicht problemlos ausgeschöpft werden. Da Reconcile quelloffen ist, ist es möglich sich durch den Quellcode zu arbeiten, um eine ausreichende Konfigurationsmöglichkeit zu finden - dies wird allerdings noch zusätzlich durch mangelhafte Kommentierung erschwert. Die Kommandozeilenargumente sehen es vor, eine externe Konfiguration vorzuschlagen. Es stellte sich aber heraus, dass es sich hierbei nur um einen Platzhalter handelt.

Viele der Vorvararbeitungsschritte sind dementsprechend auch nicht ersichtlich. Die Anmeldung auf der Mailingliste der Reconcile-Website schlug fehl: Es erfolgte eine Ablehnung, weil diese seltsamerweise nur für interne Entwickler gedacht ist. Es sei noch einmal hervorgehoben, dass nahezu alle genannten Vorteile von Reconcile unter der fehlenden Dokumentation leiden.
