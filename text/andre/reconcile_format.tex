Das Reconcileausgabeformat ermöglicht es mit einem leicht modifizierten XML-Parser die Koreferenzierungen zu extrahieren. Diese teilweise rekursiven Strukturen besitzen zwei Attribute: Eine eindeutige Identifizierungsnummer und einen Verweis auf das Kopfelement in Form von deren Identifizierungsnummer. Handelt es sich bei dem Element um einen Kopf, sind beide Zahlen identisch. Die gewünschte Darstellungsform der dritten Gruppe lässt sich ausdrücken als quasi transitive Hülle für die Koreferenzköpfe. Mit dieser Betrachtung konnten die Ketten aus der Ausgabedatei extrahiert und zusammengeführt werden.

Da in der benutzten Reconcilekonfiguration bereits die \emph{OpenNLP} Werkzeuge verwendet wurden, konnte man davon ausgehen, dass sowohl die Paragraphen- und Satzeinteilung, wie auch die Tokenisierung, identisch mit der Ausgabe der ersten Gruppe ist. Das Problem war allerdings, dass viele der benötigten Informationen nicht mehr in der Ausgabe von Reconcile enthalten waren. Es gab keine Informationen über die Nummer des Satzes oder der Token oder den Kopf der \emph{mention}. Da es nicht ersichtlich war wie Reconcile die Ausgabe gestaltet, mussten diese Informationen für die Umwandlung noch nachträglich berechnet werden.

Die Umformung läuft in folgenden Schritten ab. Zuerst werden die Sätze der Reconcileausgabe nummeriert. Für jeden Satz werden mittels eines XML-Parsers die ausgezeichneten Koreferenzen extrahiert und mit der Information über die Satznummer ergänzt. Für jeden Satz im Orginaltext werden die Token nummeriert. Diese Information wird für die Ketten ergänzt. Nun ist es möglich, die Tokennummer des Phrasenkopfes berechnen zu lassen. Von jeder Koreferenz wird danach deren transitive Hülle gebildet. Diese Datenstruktur besitzt jetzt alle nötigen Informationen für die Schnittstelle zur dritten Gruppe. Die Datenstruktur wird abschließend in XML umgewandelt.
