% !TEX encoding = UTF-8
Das Werkzeug \emph{DCoref} ist ein deterministisches Modul zur Koreferenzresolution und ist Teil der \emph{Stanford CoreNLP}. Für die Verwendung dieses Moduls ist vorausgesetzt, dass andere Module vorher auf die Eingabedaten angewendet worden sind \autocite[]{chris_stanford_dcoref}. Die Module für Wortart- und Lemmabestimmung sowie Named Entitiy Recognition und für einen Parser müssen als sogenannte Annotatoren beim Verwenden der \emph{Stanford CoreNLP} wie in Listing \ref{dcoref:required_annotators} angebeben werden.

\begin{lstlisting}[label=dcoref:required_annotators, language=Java, caption=Angabe vorausgesetzter Annotatoren in Java]
//	configure properties for pipeline of the Stanford CoreNLP
Properties properties = new Properties();
properties.setProperty("annotators", 
		"pos, lemma, ner, parse, dcoref");
//	[...] set more properties and instantiate pipeline
StanfordCoreNLP pipeline = new StanfordCoreNLP(properties);
\end{lstlisting}

% ist deterministisches Modul der Stanford CoreNLP
%hasngt von anderen Modulen ab (pos, lemma, ner, parse)
%diese Module msussen in der Pipeline vorher angewendet werden 
%arbeitet mit Sieben
%erste Stufe: bevorzugt Recall (detection)
%zweite Stufe: Siebe bevorzugen Precision
%Post-Processing: mehr Precision (?)