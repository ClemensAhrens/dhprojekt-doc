% !TEX encoding = UTF-8
Die Erstellung des Index erfolgt durch die Verarbeitung von Dateien wie in Listing\ref{output:decoref:formatiert} dargestellt.
Hierbei wird das Attribut \lstinline[language=XML]{representative="true"} verwendet, 
um die entsprechenden Koreferenzen zusammenzufassen.

Außerdem wird der Output den Bedürfnissen von Gruppe 3 angepasst.
Das abschließende Ausgabeformat hat die Form, wie sie in Listing \ref{output:buildIndex} zu sehen ist.

\begin{lstlisting}[label=output:buildIndex, name=index.xml, language=xml, caption={XML, Formatierte Ausgabe von \emph{BuildIndex}}]
<?xml version="1.0" encoding="UTF-8"?>
<root>
  <chains>
    <chain text="these two">
      <coreference>
        <id>38</id>
        <chapter>37</chapter>
      </coreference>
      <coreference>
        <id>86</id>
        <chapter>5</chapter>
      </coreference>
   ...
</root>
\end{lstlisting}

% mainklasse
%   
%   I/O Parser
%   
%   Usage msg
%   
%   call InputAnalyzer for each input.xml
%   
%   create map for coreferences with Coreference objects
%   
%   pass to OutputWriter
%   
%   call OutputWriter
% 
% \subsubsection{Coreference.java}
% class def of Coreference objects
%   
%   attr.s text chapId corefId
% 
% \subsubsection{InputAnalyzer.java}
% extract coref information
%   
%   create Map of Coreference object
%   
%     analyze input.xmls with jdom
%     
%     extract chapterID, corefID, representative mention text-element
%     
%     create Coreference object 
%     
%       ->pass to Map of Coreference object
% 
% \subsubsection{OutputWriter.java}
% multimap with representative text as key and Coreference objs
% 
%   create ArrayList if key not already existent
% 
%   add coreference for key to ArrayList in multiMap for key
% 
%   create output element
%   
%     write basic elements
%     
%     create sub elements for each coref in outputdoc
%   
% Write the complete result document to output XML file
